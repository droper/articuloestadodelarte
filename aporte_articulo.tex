El an\'alisis fundamental ha servido como base para la elaboraci\'on de modelos que identifican acciones cuyo precio va a incrementarse en el mediano plazo. Existen dos tipos de acciones consideradas por el an\'alisis fundamental: acciones de valor y acciones de crecimiento, donde las primeras son las acciones de las empresas que a juicio del inversor han sido subvaloradas por el mercado y la segundas corresponden a las empresas a las que el mercado atribuye altas tasas de crecimiento en el futuro y por las que paga elevados múltiplos respecto a su valor en libros.\\

\subsection{Modelos basados en indicadores}

Los modelos basados en indicadores utilizan los estados financieros para encontrar ratios que sirven para poder clasificar las empresas de acuerdo a sus posibilidades de aumentar su precio en el futuro. La calificación se realiza utilizando estad\'istica, eligiendo en un punto del pasado de acuerdo a los indicadores una cartera de acciones y luego evaluando su desempeño en un periodo de tiempo, generalmente de uno a cinco años.\\

Los modelos se dividen en tres tipos:\\

\subsubsection{Modelos orientados a acciones de valor}

Los modelos propuestos por Piotroski \cite{Piotroski2000}, Elleuch \cite{Elleuch2009}, Dyna Seng y Jason R. Hancock \cite{DynaSeng2012} entre otros emplean una estrategia de valor, la cual consiste en escoger acciones subvaloradas por el mercado para esperar a que el mercado reconozca su valor real y realizar las ganancias obtenidas.\\

Para hallar las acciones subvaloradas, se escogen las que cotizan con un bajo múltiplo entre el precio de mercado y el valor real. Se analizan los fundamentos de las empresas para obtener su valor real y decidir las operaciones a realizar en funci\'on de ese valor comparado con el precio de mercado.\\

Para predecir el comportamiento futuro de la empresa, se escoge un conjunto de indicadores financieros. A cada indicador se le asigna una variable la cual es 1 o 0 dependiendo si el c\'alculo del indicador arroja un resultado positivo o negativo para el futuro de las utilidades. Una vez obtenidas todas las variables, se suman, y el valor obtenido es la calificaci\'on de la acci\'on, a mayor calificaci\'on mayores son las probabilidades de que la acci\'on obtenga resultados positivos.\\

Las variables más utilizadas por las investigaciones orientadas a las acciones de valor son la variación del retorno de los activos, seguida por retorno de la inversión, varianza del inventario, cuentas por cobrar, fuerza de trabajo en relación a las ventas, la varianza del margen bruto menos la varianza ventas, la varianza en el margen bruto y la variación de la deuda de corto y largo plazo.\\

La estrategia por valor es rentable debido a que obtiene ganancias a partir de los errores del mercado en valorar las acciones, adquiriendo acciones subvaloradas y esperando a que recuperen su valor para venderlas.\\

\subsubsection{Modelos orientados a acciones de crecimiento}
	
Los modelos de Mohanram \cite{Mohanram2005} y Meena Sharma y Preeti \cite{Preeti2009} emplean una estrategia de crecimiento, por la cual escogen empresas cuyas perspectivas de crecimiento son consideradas elevadas por el mercado con el objetivo de obtener altas tasas de retorno.\\

Las empresas de crecimiento tienen precios de mercado con m\'ultiplos elevados respecto a su valor real. Para procurar diferenciar entre las empresas de crecimiento que si van a obtener rendimientos superiores a la media y las que van tener rendimientos pobres o incluso van a disminuir su valor, se analizan los fundamentos de la empresa con el fin de encontrar su valor real.\\

Para predecir si la empresa va a tener resultados positivos o no, se escogen indicadores financieros orientados a la estabilidad econ\'omica e indicadores que indiquen el potencial de crecimiento. Definidos los indicadores, a cada uno se le asigna una variable que puede ser 1 o 0 dependiendo si el cálculo del indicador arroj aun resultado positivo o negativo para el futuro de las utilidades. Obtenidas todas las variables, se suman y el resultado es la calificación de la acción, a mayor valor, mayor es la probabilidad de obtener retornos positivos.\\

Las variables más utilizadas por los modelos orientados a las acciones de crecimiento son las inversiones de capital comparadas con el sector,  la inversión en Investigación y desarrollo comparada con el sector y la inversión en marketing comparada con el sector así como la estabilidad del crecimiento.\\

Las pruebas estadísticas muestran que las acciones de crecimiento no obtienen en promedio retornos superiores al mercado. Las altas expectativas conducen a precios altos de la acción, por lo que necesitan tasas de crecimiento muy elevadas de la empresas para superar el precio de la acción.\\

Los modelos orientados a las acciones de crecimiento son más eficientes prediciendo las acciones que van a tener bajos rendimientos que las acciones cuyos rendimientos serán positivos.\\

\subsubsection{An\'alisis fundamental implementado mediante algoritmos de aprendizaje de m\'aquina}

Para llevar a cabo el an\'alisis de una acci\'on, los inversores tradicionalmente se han basado en los estados financieros tal como enseña Benjamin Graham en el libro Security Analysis \cite{BenjaminGraham2009}, pero modernas investigaciones han aplicado algor\'itmos de aprendizaje de m\'aquinas (machine learning) al an\'alisis fundamental para invertir en el mercado de valores.\\

Los modelos basados en el an\'alisis fundamental que utilizan algoritmos de aprendizaje de m\'aquinas se basan en modelos previos como el de Piotroski \cite{Piotroski2000} o Aby et. al \cite{Aby2001} para definir las variables de entrada. La variable de salida es un n\'umero en un rango que indica la tendencia a subir de precio de una acci\'on determinada y la m\'etrica m\'as utilizada es el aumento del valor de la cartera en un periodo de tiempo, por lo general un año.\\

Las inversiones realizadas utilizando aprendizaje de m\'aquinas y an\'alisis fundamental han obtenido resultados mixtos seg\'un las investigaciones analizadas, por lo que a\'un no hay un consenso sobre su grado de efectividad.\\
